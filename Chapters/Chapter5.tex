% Chapter Template

\chapter{Conclusiones} % Main chapter title

\label{Chapter5} % Change X to a consecutive number; for referencing this chapter elsewhere, use \ref{ChapterX}


%----------------------------------------------------------------------------------------

%----------------------------------------------------------------------------------------
%	SECTION 1
%----------------------------------------------------------------------------------------

\section{Conclusiones generales }

Se desarrolló e implementó satisfactoriamente un equipo casi autónomo capaz de caracterizar transformadores de baja tensión. Para llevar a cabo el trabajo se partió de requerimientos consensuados con el cliente Iris Tecnología S.R.L.. El desarrollo y las pruebas necesarias para la obtención del equipo final fueron realizadas siguiendo la planificación inicial y todos los requerimientos pudieron ser cumplidos con éxito.

Se puede destacar que no solo se cumplieron con los requerimientos iniciales del cliente, sino que, además, se incluyeron características adicionales como un mecanismo que posibilita ver los valores medidos en el \textit{display} durante el ensayo y brinda el acceso a un método de calibración de las entradas analógicas. Esto facilita el mantenimiento del equipo.

Algunos de los factores que llevaron al éxito del trabajo fueron:
\begin{itemize}
\item La temprana definición de requerimientos claros con el cliente.
\item La utilización de módulos de hardware ya armados que resolvían cuestiones de desarrollo complejas y que no estaban emparentadas con la carrera. Los sensores de corriente y tensión son un ejemplo de esto.
\item Utilizar una placa universal en vez de desarrollar y fabricar una placa de circuito impreso tradicional. Esto aceleró enormemente el desarrollo y pruebas de software.
\end{itemize}

Conocimientos adquiridos en la carrera aplicados al trabajo:
\begin{itemize}
\item Gestión de proyectos: la elaboración del plan de proyecto ayudó a organizar y pulir los requerimientos para luego dar lugar a un trabajo realizable en tiempo y forma.
\item Protocolos de comunicación: en esta materia se desarrolló el módulo utilizado para el control del \textit{display}. Además, se presentó una introducción al protocolo Wi-Fi que fue de gran utilidad.
\item Ingeniería de software en sistemas embebidos: proporcionó el conocimiento para desarrollar la arquitectura de software utilizada en el trabajo, así como el uso de Git y Doxygen.
\item Sistemas operativos de tiempo real 1 y 2: fueron de las materias más importantes para el desarrollo del trabajo. En estas materias se estudió el desarrollo de aplicaciones sobre FreeRTOS que es la base sobre la cual se desarrolló todo el proyecto.
\item Testing de software embebido: fue de gran utilidad para la realización de las pruebas de integración por medio de la metodología \textit{State transition testing} (STT).
\item Diseño de circuitos impresos: aportó al uso de la herramienta libre KiCad para el diseño del esquemático de la placa universal y los criterios básicos para la distribución de componentes utilizados en la placa universal.
\item Desarrollo de aplicaciones en sistemas operativos de propósito general: se estudió la interacción con una API REST a través del protocolo HTTP, esto fue de vital importancia ya que el servidor web del cliente está basado en esta arquitectura.
\end{itemize}

%----------------------------------------------------------------------------------------
%	SECTION 2
%----------------------------------------------------------------------------------------
\section{Próximos pasos}

Este dispositivo fue concebido por parte del cliente como una prueba de concepto para una herramienta que permita mejorar y acelerar los ensayos en transformadores. Una vez que el cliente logre validar y perfeccionar su procedimiento con esta nueva herramienta, el próximo paso sería pasar de la placa universal a una placa de circuito impreso que incluya todos los módulos de hardware. Esto le proporcionaría al equipo una mayor robustez además de ser menos voluminoso.

Otro paso importante sería la mejora de las herramientas de calibración incluidas en el proyecto, las cuales deberían estar implementada en línea con el nuevo hardware desarrollado.